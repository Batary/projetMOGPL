\documentclass[10pt,a4paper]{article}
\usepackage[utf8]{inputenc}
\usepackage[francais]{babel}
\usepackage[T1]{fontenc}
\usepackage{amsmath}
\usepackage{amsfonts}
\usepackage{amssymb}
\usepackage{graphicx}
\begin{document}
\section{Raisonnement par la programation dynamique}
\subsection{Première étape}
\noindent \textbf{Q1} \\
Si on a calculé tous les $T(i,j)$, la case $T(m-1, k)$ nous indique si on peut colorier les $m$ premières cases de la ligne $l_i$ (donc la ligne entière) avec $k$ premiers blocs (i.e. la séquence entière). \\ \\
\noindent \textbf{Q2}
\begin{itemize}
\item[1.] Si $l = 0$ et $j \in \lbrace 0, \hdots, m-1 \rbrace$, alors $T(j,l) = 0$. En effet, on peut "colorier" n'importe quel nombre de cases avec aucun bloc. \\ 
\item[2.] On suppose maintenant $l \geq 1$.
	\begin{itemize}
		\item[(a)] $j < s_l -1 \Rightarrow T(j,l) = FALSE$. En effet, cette inégalité signifie que le nombre de cases à colorier ($j+1$) est strictement plus petit que la longueur du dernier bloc. On peut pas donc colorier les $j+1$ premières cases avec le bloc $s_l$, et alors non plus avec la sous-séquence des blocs $(s_1, \hdots, s_l)$. 
		\item[(b)] $j = s_l -1 \Leftrightarrow j+1 = s_l$, ce qui signifie que la longueur du dernier bloc est exactement égale au nombre de cases à colorier. On en déduit que $T(j,1) = TRUE$ et $T(j,l) = FALSE$ pour $l > 1$. 
	\end{itemize}
\end{itemize}
\noindent
\textbf{Q3} \\
On considère dans cette question le dernier cas non traité, c'est à dire le cas où $l \geq 1, j > s_l -1$. Il y a deux possibilités: 
\begin{itemize}
	\item Soit la case $(i,j)$ restera blanche après la coloration, dans quel cas $T(j,l) = T(j-1,l)$. 
	\item Soit la case $(i,j)$ sera noire après la coloration, ce qui signifie que le bloc $s_l$ se termine à la case $(i,j)$. On en déduit qu'il commence à la case $(i, j - (s_l -1))$. Les blocs étant séparés par au moins une case blanche, la case $(i, j-s_l)$ sera blanche. Si $j-s_l > 0$, alors $T(j, l) = T(j-s_l -1, l-1)$. Si $j-s_l = 0$, alors $T(j,l) = TRUE$ si et seulement si $l = 1$.  
\end{itemize}  
\subsection{Généralisation}
\noindent\textbf{Q5} \\
\begin{itemize}
	\item[1.] Si $l = 0$ et $j \in \lbrace 0, \hdots, m-1 \rbrace$: On peut colorier $j+1$ premières cases avec 0 blocs si aucune de cases $0, \hdots, j$ n'est pas déjà coloriée à noir (ce qui imposerait une présence d'un bloc, ou au moins de sa partie). \\
	\item[2.] On suppose maintenant $l \geq 1$.
	\begin{itemize}
		\item[(a)] $j < s_l - 1 \Rightarrow T(j,l) = FALSE $ pour le même raison que précédamment.
		\item[(b)] $j = s_l$ \textbf{A FINIR PLUS TARD!!! (J'ai la flemme...)}
	\end{itemize}
\end{itemize}
\noindent 
\textbf{Q8}
\\
\noindent
Voici le tableau des temps de résolution pour les instances 1-10: \\ \\
\begin{tabular}{|c|c|}
\hline
numéro d'instance & temps de résolution [s]\\
\hline
\hline
1  & 0.013\\
\hline
2 & 5.8\\
\hline
3 & 4.2\\
\hline
4 & 10.7\\
\hline
5 & 7.7\\
\hline
6 & 22.7\\
\hline
7 & 10.7\\
\hline
8 & 18.5\\
\hline
9 & 342.7\\
\hline
10 & 349.2\\
\hline
\end{tabular}
\newpage
\section{La PLNE}
\subsection{Modélisation}
\noindent
\textbf{Q10} \\
\noindent
Soit $l_i$ la i-ième ligne avec une séquence associée $(s_1, \hdots ,s_k)$. Si le bloc $t$ de longueur $s_t$ commence par la case $(i,j)$, alors les cases $(i,j)$ à $(i,j+s_t-1)$ doivent être noires, ce qui s'exprime comme: $$ y_{ij}^{t} \leq \frac{\sum_{l = j}^{j+s_t-1} x_{il}}{s_t}$$\\ 
\noindent
De manière analogue, on a pour la j-ième colonne $c_j$ possédant la séquence $(s'_{1}, \hdots , s'_{k'})$ : 
$$ z_{ij}^{t} \leq \frac{\sum_{l = i}^{i+s'_t-1} x_{lj}}{s_t}$$
\textbf{Q11} \\
Avec les notations de la question précédente, on souhaite d'exprimer le fait que si le bloc $t$ de la i-ième ligne commence à la case $(i,j)$, alors le $(t+1)$-ième bloc ne peut pas commencer avant la case $(i, j+ s_t +1)$. Ce qui se formule par: 
$$ y_{ij}^t \leq \sum_{l = j+s_t+1}^N y_{il}^{t+1} , t \in \lbrace 1, \hdots, k-1 \rbrace$$
\noindent
De manière analogue, on obtient pour les colonnes: 
$$ z_{ij}^t \leq \sum_{l = j+s_t+1}^M y_{lj}^{t+1} , t \in \lbrace 1, \hdots, k'-1 \rbrace$$
\textbf{Q12} \\
\subsection{Implantations et tests}
\noindent
\textbf{Q13}
\end{document}